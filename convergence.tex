\section{State convergence}
\label{sec:crdt}
In the previous section we discussed how RedBlue consistency achieves state convergence by
relying on shadow operations that commute with each other.
With this approach, defining a new operation also implies
writing one or more commutative shadow operations, each of which corresponds to a distinct side effect.
The major challenge of doing this manual work
is that, in an application with a large number of operations, this process may be
complex and error-prone.

We now discuss an alternative principled approach to create commutative operations by design.
Our approach builds on conflict-free replicated data types (CRDTs)~\cite{crdts}, which are
specially-designed data structures that can be replicated and modified concurrently,
and include mechanisms to merge concurrent updates in a deterministic way.
Application operations consist of updates to these elementary data types, thus
guaranteeing state convergence.

\subsection{CRDTs}

A CRDT is a data type that can be replicated at multiple replicas. As such,
it defines an interface with a set of operations to read and to modify its state.
A CRDT replica can be modified by locally executing an update operation.
When different replicas of the same object are modified concurrently, they
temporarily diverge.
CRDTs have built-in support for achieving strong eventual consistency~\cite{crdts}, in which
all replicas will eventually reach the same (equivalent) state after applying
the same set of updates, without relying on a distributed conflict arbitration
process.

Two main flavors of CRDTs have been studied in the literature: operation-based CRDTs
and state-based CRDTs. For each of these, sufficient conditions for
achieving strong eventual consistency have been established.

In operation-based CRDTs (or commutative replicated data types), updates
are propagated by broadcasting operations to every replica in causal order.
Interestingly, this proposal matches the operation execution decomposition presented
in RedBlue consistency (Section~\ref{sec:redblue}), where operations are divided in two components, a {\em generator}
operation that executes in the local replica, has no side effect and produces a
{\em shadow} operation, which is propagated and executed in all replicas. 
The two types of operations are analogous to {\em prepare} and {\em downstream} operations
in the context of  operation-based CRDTs, respectively, with the main difference that shadow operations are assigned a consistency level in RedBlue consistency.
Similarly to the consequence of commutative shadow operations, the replicas of an operation-based CRDT converge to the same state after executing the same set of updates (in any order that respects causality)
if the execution of any two concurrent downstream operations commutes~\cite{crdts}.

A state-based CRDT (or convergent replicated data type) defines, in addition to
the operations to read and update its state, an operation to merge the
state of two replicas.
Replicas synchronize by exchanging the full replica states: when a new
state is received, the new updates are incorporated in the local replica
by executing the merge function.
It has been shown that the replicas of a state-based CRDT converge to the same state
after all replicas synchronize (directly or indirectly) if:
\begin{inparaenum}[(1)]
\item all the possible states of an object are partially ordered, forming a
join-semilattice;
\item the merge operation between two states is the semilattice join; and
\item an update monotonically increases the state according to the defined
partial order~\cite{crdts}.
\end{inparaenum}

\subsection{Examples}

CRDTs have been used in a number of research systems, such as Walter \cite{walter}
and SwiftCloud \cite{Zawirski15Write},
and commercial systems, such as Riak \cite{riak} and SoundCloud \cite{soundcloud}.
These systems include CRDTs that implement several data types, such as
registers, counters, sets, maps, and flags. For each such data type, it is possible to define and implement different semantics
to handle concurrent updates, leading to different CRDTs.
These semantics define which is the final state of a CRDT when
concurrent updates occur.
For example, for sets, it is possible to define an \emph{add-wins} semantics,
where, in the presence of a concurrent add and remove of some element $e$,
the final state will contain $e$ (or, more precisely, there exists an add of $e$
that does not happen before a remove of $e$).
It is also possible to define a \emph{remove-wins} semantics, where the remove will
win over a concurrent add.
Other semantics can also be implemented, such as a \emph{last-writer-wins} strategy where
an element will belong to the set or not depending on which was the last operation
executed, according to the order among operations.

When creating an application, an application developer must select the CRDT
with the most appropriate semantics for its goal.
For example, in the bank account example, the balance of an account can be modeled
as a counter and the set of accounts of a client can be maintained in an add-win set or map CRDT.

In general, an application operation will manipulate multiple data objects.
When using CRDTs, it is possible to maintain replicas of these objects in multiple nodes.
An operation can execute by accessing a single replica of each object it accesses.
These updates can later be propagated to other nodes, with CRDT rules guaranteeing
that the replicas of each object will converge to the same state.
By propagating the updates to all objects modified in an operation atomically,
it is possible to guarantee that all effects of an operation are observed at the
same time.

\paragraph{CRDTs for relational databases}

\begin{table}[t!]
\small
\centering
\begin{tabular}{|c|c|p{9.3cm}|}
\hline
SQL type & CRDT & Description \\
\hline
\hline
\multirow{2}{*}{FIELD*} & LWW &  Use last-writer-wins to solve concurrent updates\\\cline{2-3}
 & NUMDELTA  &  Add a delta to the numeric value \\\hline
\multirow{2}{*}{TABLE} & AOSET,                       UOSET,       & Sets with restricted operations (add, update, and/or remove). \\
& AUSET,              ARSET & Conflicting operations are logically executed by timestamp order.\\
\hline
\end{tabular}
\caption{Commutative replicated data types (CRDTs) for relational data. *
FIELD covers primitive types such as integer, float, double, datetime and string.}
\label{tab:crdts}
\end{table}

In relational databases, it is also possible to model data using CRDTs.
Table \ref{tab:crdts} presents the mapping proposed in SIEVE \cite{Li14Automating}.
Regarding table fields, we defined only two CRDTs. The \emph{LWW} CRDT can be
used with any field type and implements a \emph{last-writer-wins} strategy
for defining the final value of a field.
The \emph{NUMDELTA} CRDT can be used with numeric fields, and transforms each
update operation in a downstream operation that adds or subtracts
a constant to the value of the field.
This can be used to support account balances,
counters, etc.

A database table can be seen as a set of tuples.
In the general case, and following the semantics of the \emph{ARSET} CRDT, when concurrent
\emph{insert}, \emph{update} and \emph{delete} operations occur,
the following rules can be used:
\begin{inparaenum}[(1)]
\item concurrent inserts of tuples with the same key are treated as an insert followed
by a sequence of updates;
\item for concurrent updates, the rules defined for fields
are used to deterministically define the final value;
\item a delete will only take effect if no concurrent update or insert was executed.
\end{inparaenum}

While using CRDTs guarantees that all replicas converge to the same state, it does
not guarantee that the convergence rules executed independently by different CRDTs
maintain application invariants. Next, we show how we can address
this problem by restricting the concurrent execution of operations that can break
application invariants.
%, \hl{instead of serializing all non-invariant safe shadow operations.}



%The implementation of these CRDTs in a relational database is detailed
%elsewhere \cite{Li14Automating}.
%We note that although both CRDTs ensure convergence, their semantics
%differ significantly.
%For example, the first CRDT leads to a final state that does not
%reflect the effects of all update operations, while the second is only
%correct if all updates are commutative.
%
%For implementing these CRDTs in a relational database, we have added additional
%metadata to each table for recording for each field/row the timestamp of the
%last update and of the last delete operation.
%The information of a deleted row is not removed until it is known that no
%concurrent update was executed.
%
%We have to additionally modify the SQL operations to:
%\begin{inparaenum}[(1)]
%\item ignore deleted rows in select operations;
%\item update the metadata on insert, update and delete operations;
%\item use the metadata to implement to appropriate CRDT semantics.
%\end{inparaenum}




%Next, we will show how we can define CRDTs that are appropriate for handling
%concurrent updates in relational databases.

%While this approach guarantees that all replicas converge to the same state, it does
%not guarantee that the convergence rules executed independently by different CRDTs
%maintain application invariants. In Section~\ref{sec:indigo}, we show how we can address
%this problem by restricting the concurrent execution of operations that can break
%application invariants.
%
%\subsection{CRDTs for relational databases}
%
%We now show that data in a relational database can be modeled using
%CRDTs.
%Table \ref{tab:crdts} presents the mapping proposed in SIEVE \cite{Li14Automating}.
%
%\begin{table}[t!]
%\footnotesize
%\centering
%\begin{tabular}{|c|c|p{8cm}|}
%\hline
%SQL type & CRDT & Description \\
%\hline
%\multirow{3}{*}{FIELD*} & LWW &  Use last-writer-wins to solve concurrent updates\\\cline{2-3}
% & NUMDELTA  &  Add a delta to the numeric value \\\hline
%TABLE & AOSET,                       UOSET,                AUSET,              ARSET & Sets with restricted operations (add, update, and/or remove).
%Conflicting ops.\ are logically executed by timestamp order.\\
%\hline
%\end{tabular}
%\caption{Commutative replicated data types (CRDTs) for relational data. *
%FIELD covers primitive types such as integer, float, double, datetime and string.}
%\label{tab:crdts}
%\end{table}
%
%A database table can be seen as a set of tuples (with the key identifying
%the tuple).
%We have defined different semantics for these sets, that whether
%concurrent \emph{insert} and \emph{delete} SQL operations are allowed
%and how they are treated.
%For example, the \emph{AOSET} does not support \emph{delete} operations.
%The \emph{ARSET} supports concurrent \emph{insertion}, \emph{updates} and
%\emph{deletes}.
%Concurrent insertions of tuples with the same key are logically transformed
%in an insert followed by an update.
%For concurrent updates of the same data, the rules defined for fields
%are used to deterministically define the final value.
%A delete operation will only take effect if no concurrent update or insert
%is executed. Thus, the ARSET provides a semantics that can be characterized as
%update-wins, where any update to data wins over a concurrent delete.
%
%For table fields we have defined only two CRDTs. The \emph{LWW} CRDT can be
%used with any field type and implements a \emph{last-writer-wins} strategy where
%the update with the highest timestamp is selected for defining the final value of a field.
%
%For numeric fields, when a field is updated using a SQL update command,
%the programmer could have two different
%intentions in terms of what the update means and how concurrent
%updates should be handled:
%\begin{inparaenum}[(i)]
%\item the update can represent a delta to be
%added or subtracted from the current value (e.g., when updating the
%stock of a certain item), in which case all concurrent updates should
%be applied possibly in a different order at all replicas to ensure that no stock
%changes are lost, or
%\item it can be overwriting an old value with a new
%value (e.g., when updating the year of birth in a user profile), in
%which case an order for these updates should be arbitrated, and the last
%written value should prevail.
%\end{inparaenum}
%The latter case can be addressed using the \emph{LWW} CRDT, while
%for the former we have defined the \emph{NUMDELTA} CRDT, where each
%update operation is transformed in a downstream operation that adds or subtracts
%a constant to the value of the field.
%
%Both strategies ensure convergence, but their semantics differ significantly.
%For example, the second strategy leads to a final state that does not
%reflect the effects of all update operations, while the second is only
%correct if all updates are commutative.
%
%For implementing these CRDTs in a relational database, we have added additional
%metadata to each table for recording for each field/row the timestamp of the
%last update and of the last delete operation.
%The information of a deleted row is not removed until it is known that no
%concurrent update was executed.
%
%We have to additionally modify the SQL operations to:
%\begin{inparaenum}[(1)]
%\item ignore deleted rows in select operations;
%\item update the metadata on insert, update and delete operations;
%\item use the metadata to implement to appropriate CRDT semantics.
%\end{inparaenum}
%

