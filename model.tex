\section{System model}
\label{sec:model}

%\hl{Note that the following paragraph is new. If you like it, then we have to change all terms ``site'' to ``replica'' throughout the whole paper.}

Our system model is that of a
fully replicated distributed system, where replicas are located in different data centers. Each replica follows a deterministic state machine: there is a set of operations $\mathcal{U}$, which manipulate
a set of reachable system states $\mathcal{S}$. %When operation $u$ is applied against a state $S$,
%it produces another state $S'$ (denoted $S'=S+u$). 
Each operation $u$ is initially submitted at a given replica (preferably in the closest data center), which we
call the \emph{origin replica} of $u$.
When the remaining replicas receive a request to replicate this operation, they will apply the operation against their local state. 

%% \hl{remove the whole paragraph. }Our system model is that of a
%% fully replicated distributed system, where replicas are located in $k$ sites denoted by $\textit{site}_0\ldots \textit{site}_{k-1}$. Each replica follows a deterministic state machine: there is a set of operations $\mathcal{U}$, which manipulate
%% a set of reachable system states $\mathcal{S}$. %When operation $u$ is applied against a state $S$,
%% %it produces another state $S'$ (denoted $S'=S+u$). 
%% Each operation $u$ is initially submitted at a given site, which we
%% call the \emph{primary site} of $u$ and denote $\textit{site}(u)$.
%% When a replica at the remaining sites receives a request to replicate this operation, that replica applies the operation against its local state. 


Throughout our explanation we will highlight two important properties
that the replicated system should obey. First, there is
the {\bf state convergence} property, which says that all the sites
that have executed the same set of operations against the same initial state are in the same final state.
%irrespectively of the order in which those operations were executed.
This is important to prevent a situation where
the system quiesces (no more updates are received) and
read-only queries return different results
depending on which sites the users are connected to.
%Another important property that consists of ensuring a good user
%experience is to preserve \textbf{causality}, both in terms of the monotonicity of user requests
%within a session and preserving causality across clients, which is key
%to enabling natural semantics~\cite{Petersen1997Flexible}. Additionally,
%operations invoked by client requests should return a {\bf single value}, precluding solutions that return a set of values
%corresponding to the outcome of multiple concurrent updates. 
The second property is to {\bf preserve application-specific invariants}, which comprise a specification for the behavior of the system.
%The second property is a specification for the behavior of the system, stating that the system should maintain a set of %application-specific
%{\bf invariants}. 
To define these, we introduce the primitive $\textit{valid}(S)$ to be {\it
  true} if state $S$ obeys these invariants and {\it false}
otherwise. 
%This allows us to introduce a basic notion of correctness
%of the implementation of a given service: an operation $u$ is {\it correct} if, for every valid state $S$, $S + u$ is also valid. 
